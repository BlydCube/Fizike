\documentclass[a4paper, twocolumn]{article}

\usepackage[utf8]{inputenc}
\usepackage{amsmath}
\usepackage{tikz}
\usepackage{pgfplots}
\usepackage{mathtools}

\DeclarePairedDelimiter\abs{\lvert}{\rvert}

\pgfplotsset{width=7cm,compat=1.9}

\title{Permbledhje: \\ Tremujori - 2}
\author{Kristian Blido}
\date{13-03-2021}
\begin{document}
\tableofcontents
\maketitle
\addtocounter{section}{8}
\section{Fizika Termike}
\subsection{Ndryshimi i Gjendjes dhe Energjise}
\begin{tikzpicture}
	\begin{axis}[
		axis lines = left,
    		xlabel = $t$,
    		ylabel = {$T$},
		grid=both,
		grid style={line width=.1pt, draw=gray!30}
		xtick={-5, 0, 5, 10, 15, 20}
		%minor tick num=5
		]
%Below the red parabola is defined
\addplot [
    domain=-5:0, 
    samples=100, 
    color=black,
]
{x+5};
\addplot [
    domain=0:5, 
    samples=100, 
    color=black,
]
{5};
\addplot [
    domain=5:10, 
    samples=100, 
    color=black,
]
{x};
\addplot [
    domain=10:15, 
    samples=100, 
    color=black,
]
{10};
\addplot [
    domain=15:20, 
    samples=100, 
    color=black,
]
{x-5};
	\addplot[mark=*] coordinates {(-5, 0)};
	\addplot[mark=*] coordinates {(0, 5)};
	\addplot[mark=*] coordinates {(5, 5)};
	\addplot[mark=*] coordinates {(10, 10)};
	\addplot[mark=*] coordinates {(15, 10)};
	\addplot[mark=*] coordinates {(20, 15)};
\end{axis}
\end{tikzpicture}
$$\begin{array}{r c l l}
	]-5, 0[ & \to&\textrm{Ngrohje}& Q=c\cdot m \cdot \Delta T \\
	]0, 5[ &\to& \textrm{Shkrirje}& Q=\lambda \cdot m \\
	]5, 10[ &\to& \textrm{Ngrohje}& Q=c\cdot m \cdot \Delta T  \\
	]10, 15[ &\to& \textrm{Avullim}& Q=q \cdot m\\
	]15, 20[ &\to& \textrm{Ngrohje}& Q=c\cdot m \cdot \Delta T \

\end{array}
$$
\subsection{Energjia e Brendshme}
\[
U = \left\{
{\begin{array}{l l l}
		\frac{3}{2}\cdot R\cdot T\cdot n, &1&\text{atom} \\
		\\
		\frac{5}{2}\cdot R\cdot T\cdot n, &2&\text{atome}\\
		\\
		3\cdot R\cdot T\cdot n, &3+&\text{atome}
\end{array}\right.}
\]
\\
\[\begin{array}{ccccc}
	R&=&N_{A}&\cdot &k_{B}\\
	\\
	 &=&6.02\cdot 10^{23} \frac{}{mol} &\cdot & 1.38 \cdot \frac{m^2 kg}{10^{23}\cdot s^{2}\cdot K^{1}}   \\
	 \\
	 &=&8.31 \frac{m^2\cdot kg}{s^{2}\cdot K\cdot mol}\\
	 \\
	 &=&8.31 \frac{J}{mol\cdot K}
\end{array}
\]
\[
n=\frac{m}{M}=\frac{N}{N_{A}}
\] 
\[
	T(K)=T(^{\circ}C)+273.15
\] 
\section{Gazet Ideale}
\subsection{Ligji i gazeve}
\[
P\cdot V=N\cdot k_{b}\cdot T
\] 
\[
	P\cdot V=n\cdot (N_{A}\cdot k_{B})\cdot T
\] 
\[
P\cdot V=n\cdot R\cdot T
\] 
\subsection{Energjia Kinetike}
\[
{\epsilon}_{k} = \frac{3}{2} \cdot k_{B}\cdot T
\] 
\subsection{Parimi i pare i Termodinamikes}
\[
Q=\Delta U + A
\] 
"Sasia e nxehtesise qe merr nje sistem shkon pjeserisht per ndryshimin e energjise se brendshme dhe pjeserisht per kryerjen e punes"

\subsection{Izoproceset}
\subsubsection{Procesi Ciklik}
\begin{itemize}
	\item 2 rruge Termodinamike
	\item Sisteme \emph{Quasi-Statike}
\end{itemize}
\[
\begin{Bmatrix}
T_{1}&=& T_{2} \\
\Delta U &=& 0 \\
Q &=& A \\
\end{Bmatrix}
\] 
\subsubsection{Procesi Izotermik}
 \[
\frac{P_{1}}{P_{2}} = \frac{V_{2}}{V_{1}}
\] 
\[
\begin{Bmatrix}
T_{1}&=&T_{2} \\
\Delta U &=& 0 \\
Q &= &A \\
\end{Bmatrix}
\]
\subsubsection{Izobarik}
\[
\frac{V_{1}}{V_{2}} = \frac{T_{1}}{T_{2}}
\] 
\[
\begin{Bmatrix}
P_{1}&= &P_{2} \\
Q&=&\Delta U + A\\
\end{Bmatrix}
\]
\subsubsection{Izohorik}
\[
\frac{P_{1}}{T_{2}} = \frac{P_{2}}{T_{2}}
\] 
\[
\begin{Bmatrix}
V_{1}&=& V_{2} \\
A&=&0\\
Q&=&\Delta U\\
\end{Bmatrix}
\]
\subsubsection{Procesi Adiabatik}
\[
\begin{Bmatrix}
Q&=&0\\
A&=&-\Delta U\\
\end{Bmatrix}
\]
\subsection{Parimi i dyte i Termodinamikes}
"Nuk mund te ekzistoje motorri i perjetshem"
\[
A=Q_{i}-Q_{f}
\] 
\[
A=T_{i}-T_{f}
\]
$ \text{Rendimenti} \to \eta $
\[
\begin{Bmatrix}
	\eta = \frac{A}{Q_{i}}\\
	\eta < 1
\end{Bmatrix}
\begin{Bmatrix}
	\eta = \frac{A}{T_{i}}\\
	\eta < 1
\end{Bmatrix}
\] 
\section{Fusha Elektrike}
\subsection{Intensiteti i Fushes Elektrike}
\[
	E=\frac{F}{q} \left( \frac{N}{C} \right)
\] 
\subsection{Ligji i Kulonit}
\begin{eqnarray*}
\abs{\vec{F}}&=&k\cdot \frac{Q_{1}\cdot Q_{2}}{\epsilon \cdot r^2}\\
\\
&=&\frac{1}{4\cdot \pi \cdot \epsilon _{0}} \cdot \frac{Q_{1}\cdot Q_{2}}{\epsilon \cdot r^2}\\
\\
&=&\frac{Q_{1}\cdot Q_{2}}{4\cdot \pi \cdot \epsilon _{0}\cdot \epsilon \cdot r^2}\\
\end{eqnarray*}
Ku $\epsilon _{0} = 8.85\cdot 10^{-12} \frac{F}{m}$ dhe $k=9\cdot 10^9 \frac{N \; m^2}{C^{-2}}$
\subsection{Intensiteti i Fushes Elektrike Qendrore}
\begin{eqnarray*}
	E&=& \frac{F}{q} \\
	\\
	 &= & \frac{\frac{Q_{1}\cdot q}{4\cdot \pi \cdot \epsilon _{0}\cdot \epsilon \cdot r^2} }{q} \\
	 \\
	 &= & \frac{Q}{4\cdot \pi \cdot \epsilon _{0}\cdot \epsilon \cdot r^2} \\
\end{eqnarray*}
\subsection{Potenciali Elektrik}
\[
V=\frac{W_{P}}{q}
\] 
\subsection{Intensiteti i Fushes se Njetrajtshme}
\begin{eqnarray*}
	A&=&W_{P}\\
	F\cdot \Delta d &=& \Delta V \cdot q\\
	\frac{F}{q}&=& \frac{\Delta V}{\Delta d}\\
	E &=& -\frac{\Delta V}{\Delta d}\\
\end{eqnarray*}
\subsection{Potenciali i Fushes Qendrore}
\[
V=\frac{q}{4\cdot \pi \cdot \epsilon _{0} \cdot \epsilon \cdot r}
\] 
\section{Kondensatoret}
\subsection{Kapaciteti}
\[
	C=\frac{q}{V} (F)
\] 
\subsubsection{Kapaciteti i Percjellesit}
\[
C=\frac{q}{\Delta V}=\frac{q}{U}
\] 
\subsection{Energjia e Kondesatorit}
\begin{eqnarray*}
	W&=&\frac{Q\cdot V}{2}\\
	 &=&\frac{(C\cdot V)\cdot V}{2}\\
	 &=&\frac{C\cdot V^2}{2}\\
	 &=&\frac{Q^2}{2\cdot C}
\end{eqnarray*}
\subsection{Dendesia e Ngarkesave}
\[
\begin{array}{r l c l}
	\text{lineare} & \to& \lambda, & \lambda = \frac{q}{l}\\
	\\
	\text{siperfaqje} & \to& \sigma, & \sigma = \frac{q}{s}\\
	\\
	\text{vellim}& \to& \rho, & \rho = \frac{q}{v}\\
\end{array}
\] 
\subsection{Kapaciteti i Kondensatorit}
\[
	{\left\{ \begin{array}{l} E=\frac{q}{S\cdot \epsilon \cdot \epsilon_{0}}\\
				\\
		E = \frac{V}{d} 
\end{array}
\] 
\[
\frac{Q}{S\cdot \epsilon \cdot \epsilon_{0}}=\frac{V}{d}
\] 
\[
	\frac{q}{V}=C=\frac{\epsilon \cdot  \epsilon_{0} \cdot S}{d}
\] 
\subsubsection{Depertueshmeria Elektrike}
\[
\epsilon = \frac{C}{C_{0}}
\] 
\[
\epsilon_{0}=\frac{1}{\mu_{0}\cdot c}
\]
\[
\begin{array}{rcl}
	\epsilon_{0} &\to& \text{Pershkueshmeria elektrike ne vakum}\\
	\mu_{0} &\to& \text{Pershkueshmeria magnetike vakum}\\
	c &\to& \text{Shpejtesia e drietes ne vakum}

\end{array}
\] 
\subsection{Lidhja e Kondensatoreve}
\subsubsection{Ne Paralel}
\[
\begin{array}{c c l}
	C&=&\sum C_{i} \\
	\\
	\Delta V&=&V_1=V_2=V_3=\ldots=V_i\\
	\\
	q&=& \sum q_i
\end{array}
\] 
\subsubsection{Ne Seri}
\[
	\begin{array}{c c l}
		\frac{1}{C}&=&\sum \frac{1}{C_{i}}  \\
		\\
		\Delta V &=& \sum V_{i}\\
		\\
		q&=& q_1=q_2=q_3=\ldots =q_i  \\
\end{array}
\] 
\section{Rryma Elektrike}
\subsection{Rryma}
\[
	I=\frac{\Delta Q}{\Delta t}\;\;\; (A)
\] 
\subsection{Dendesia e Rrymes}
\[
J=\frac{I}{S}
\] 
\subsection{Forca Elektro Motorre}
\[
	\epsilon = \frac{A}{q}=\frac{q\cdot V}{q}=\Delta V
\] 
\subsection{Rezistenca Elektrike}
\[
R=\rho \cdot \frac{l}{S}
\] 
\subsection{Ligji i Ohmit}
\[
	I=\frac{\epsilon}{R+r}
\] 
\subsection{Ligji i Joul-Lencit}
\[
Q=I^2\cdot R\cdot \Delta t
\] 
\subsection{Fuqia Elektrike}
\begin{eqnarray*}
	P&=&\frac{W}{\Delta t}\\
	\\
	 &=&\frac{V\cdot \Delta Q}{\Delta T}\\
	 \\
	 &=&V\cdot I\\
	 \\
	 &=&I^2 \cdot R\\
	 \\
	 &=&
\end{eqnarray*}
\[
x \elem \R
\] 
\end{document}
