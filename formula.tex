\documentclass[a4paper, twocolumn]{article}

\usepackage[utf8]{inputenc}
\usepackage{amsmath}
\usepackage{tikz}
\usepackage{pgfplots}

\pgfplotsset{width=7cm,compat=1.9}

\title{Permbledhje: \\ Tremujori - 2}
\author{Kristian Blido}
\date{13-03-2021}
\begin{document}
\tableofcontents
\maketitle
\addtocounter{section}{8}
\section{Fizika Termike}
\subsection{Ndryshimi i Gjendjes dhe Energjise}
\begin{tikzpicture}
	\begin{axis}[
		axis lines = left,
    		xlabel = $t$,
    		ylabel = {$T$},
		grid=both,
		grid style={line width=.1pt, draw=gray!30}
		xtick={-5, 0, 5, 10, 15, 20}
		%minor tick num=5
		]
%Below the red parabola is defined
\addplot [
    domain=-5:0, 
    samples=100, 
    color=black,
]
{x+5};
\addplot [
    domain=0:5, 
    samples=100, 
    color=black,
]
{5};
\addplot [
    domain=5:10, 
    samples=100, 
    color=black,
]
{x};
\addplot [
    domain=10:15, 
    samples=100, 
    color=black,
]
{10};
\addplot [
    domain=15:20, 
    samples=100, 
    color=black,
]
{x-5};
	\addplot[mark=*] coordinates {(-5, 0)};
	\addplot[mark=*] coordinates {(0, 5)};
	\addplot[mark=*] coordinates {(5, 5)};
	\addplot[mark=*] coordinates {(10, 10)};
	\addplot[mark=*] coordinates {(15, 10)};
	\addplot[mark=*] coordinates {(20, 15)};
\end{axis}
\end{tikzpicture}
$$\begin{array}{r c l l}
	]-5, 0[ & \to&\textrm{Ngrohje}& Q=c\cdot m \cdot \Delta T \\
	]0, 5[ &\to& \textrm{Shkrirje}& Q=\lambda \cdot m \\
	]5, 10[ &\to& \textrm{Ngrohje}& Q=c\cdot m \cdot \Delta T  \\
	]10, 15[ &\to& \textrm{Avullim}& Q=q \cdot m\\
	]15, 20[ &\to& \textrm{Ngrohje}& Q=c\cdot m \cdot \Delta T \

\end{array}
$$
\subsection{Energjia e Brendshme}
\[
U = \left\{
{\begin{array}{l l l}
		\frac{3}{2}\cdot R\cdot T\cdot n, &1&\text{atom} \\
		\\
		\frac{5}{2}\cdot R\cdot T\cdot n, &2&\text{atome}\\
		\\
		3\cdot R\cdot T\cdot n, &3+&\text{atome}
\end{array}\right.}
\]
\\
\[\begin{array}{ccccc}
	R&=&N_{A}&\cdot &k_{B}\\
	\\
	 &=&6.02\cdot 10^{23} \frac{}{mol^{-1}} &\cdot & 1.38 \cdot \frac{m^2 kg}{10^{23}\cdot s^{2}\cdot K^{1}}   \\
	 \\
	 &=&8.31 \frac{m^2\cdot kg}{s^{2}\cdot K\cdot mol}\\
	 \\
	 &=&8.31 \frac{J}{mol\cdot K}
\end{array}
\]
\[
n=\frac{m}{M}=\frac{N}{N_{A}}
\] 
\[
	T(K)=T(^{\circ}C)+273.15
\] 
\subsection{Energjia Kinetike}
\[
{\epsilon}_{k} = \frac{3}{2} \cdot k_{B}\cdot T
\] 
\subsection{Ligji i gazeve}
\[
P\cdot V=N\cdot k_{b}\cdot T
\] 
\[
	P\cdot V=n\cdot (N_{A}\cdot k_{B})\cdot T
\] 
\[
P\cdot V=n\cdot R\cdot T
\] 
\subsection{Parimi i pare i Termodinamikes}
\[
Q=\Delta U + A
\] 
"Sasia e nxehtesise qe merr nje sistem shkon pjeserisht per ndryshimin e energjise se brendshme dhe pjeserisht per kryerjen e punes"

\subsection{Izoproceset}
\subsubsection{Procesi Ciklik}
\begin{itemize}
	\item 2 rruge Termodinamike
	\item Sisteme \emph{Quasi-Statike}
\end{itemize}
\[
\begin{Bmatrix}
T_{1}&=& T_{2} \\
\Delta U &=& 0 \\
Q &=& A \\
\end{Bmatrix}
\] 
\subsubsection{Procesi Izotermik}
 \[
\frac{P_{1}}{P_{2}} = \frac{V_{2}}{V_{1}}
\] 
\[
\begin{Bmatrix}
T_{1}&=&T_{2} \\
\Delta U &=& 0 \\
Q &= &A \\
\end{Bmatrix}
\]
\subsubsection{Izobarik}
\[
\frac{V_{1}}{V_{2}} = \frac{T_{1}}{T_{2}}
\] 
\[
\begin{Bmatrix}
P_{1}&= &P_{2} \\
Q&=&\Delta U + A\\
\end{Bmatrix}
\]
\subsubsection{Izohorik}
\[
\frac{P_{1}}{T_{2}} = \frac{P_{2}}{T_{2}}
\] 
\[
\begin{Bmatrix}
V_{1}&=& V_{2} \\
A&=&0\\
Q&=&\Delta U\\
\end{Bmatrix}
\]
\subsubsection{Procesi Adiabatik}
\[
\begin{Bmatrix}
Q&=&0\\
A&=&-\Delta U\\
\end{Bmatrix}
\]
\subsection{Parimi i dyte i Termodinamikes}
"Nuk mund te ekzistoje motorri i perjetshem"
\[
A=Q_{i}-Q_{f}
\] 
\[
A=T_{i}-T_{f}
\]
$ \text{Rendimenti} \to \eta $
\[
\begin{Bmatrix}
	\eta = \frac{A}{Q_{i}}\\
	\eta < 1
\end{Bmatrix}
\begin{Bmatrix}
	\eta = \frac{A}{T_{i}}\\
	\eta < 1
\end{Bmatrix}
\] 
\end{document}
